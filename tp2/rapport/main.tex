\documentclass{article}

\usepackage{graphicx} % Required for the inclusion of images
\usepackage[french]{babel}
\usepackage{datetime}
\usepackage{lmodern}
\usepackage[utf8]{inputenc}
\usepackage[T1]{fontenc}
\usepackage{times} % Uncomment to use the Times New Roman font
\usepackage{listings}
\usepackage{booktabs}

\setlength\parindent{0pt} % Removes all indentation from paragraphs

\begin{document}

    \begin{titlepage}

    \newcommand{\HRule}{\rule{\linewidth}{0.5mm}}
    \center

    \begin{figure}[h]
        \includegraphics[scale=1,center]{poly.png}
    \end{figure}

    \small{DÉPARTEMENT DE GÉNIE INFORMATIQUE ET DE GÉNIE LOGICIEL} \\[2cm]

    \HRule \\[0.5cm]
    \huge{\bfseries INF8215}\\
    \huge{\bfseries Intelligence artif.: méthodes et algorithmes}\\[0.3cm]
    \large{RAPPORT TP2 - Recherche Adversarielle}\\[0.2cm]
    \HRule \\[1.8cm]

    Par:\\
    \large \textbf{Yoan Gauthier}\\
    \large \textbf{Adam Martin-Côté}
    \vfill

    10 novembre 2019

\end{titlepage}


    \section{Implémentation de base}

    Execution des test:

    \texttt{python3 -m unittest tp2/tests/testrushhour.py}

    \section{Implémentation d'une recherche minimax}

    \subsection{simple}

    Execution des test:

    \texttt{python3 -m unittest tp2/tests/testminimaxsingleplayer.py}

    \subsection{adversarielle}

    Execution des test:

    \texttt{python3 -m unittest tp2/tests/testminimaxtwoplayers.py}

    \vspace{0.5cm}

    L'exécution des tests génère un fichier csv pour les algorithmes Minimax, Expectimax et Élagage.
    Les résultats sont reproduit dans les tableaux suivants

    \subsection{Résultats}

    \begin{table}[!ht]
    \centering
    \begin{tabular}{@{}ll@{}}
        \toprule
        test\_name & time (ms) \\ \midrule
        test\_solve\_two\_player\_1 & 2423.07   \\
        test\_solve\_two\_player\_2 & 3728.19   \\
        test\_solve\_two\_player\_3 & 1327.92   \\ \bottomrule
    \end{tabular}
    \caption{R\'esultats avec Minimax}
    \label{tab:my-table}
\end{table}


    \section{élagage $\alpha-\beta$}

    Execution des test:
    \texttt{python3 -m unittest tp2/tests/testpruning.py}

    \begin{table}[!ht]
    \centering
    \begin{tabular}{@{}ll@{}}
        \toprule
        test\_name & time (ms) \\ \midrule
        test\_pruning\_1 & 650.46    \\
        test\_pruning\_2 & 815.28    \\
        test\_pruning\_3 & 612.69 \\ \bottomrule
    \end{tabular}
    \caption{Résultats avec élagage}
\end{table}


    \clearpage

    \section{Expectimax}

    Execution des test:
    \texttt{python3 -m unittest tp2/tests/testexpectimax.py}

    \begin{table}[!ht]
    \centering
    \begin{tabular}{@{}lll@{}}
        test\_name & time (ms) & coups \\ \midrule
        test\_expectimax\_1 & 1886.22 & 12    \\
        test\_expectimax\_2 & 8386.26 & 80    \\
        test\_expectimax\_3 & 2524.92 & 40    \\
        test\_expectimax\_optimistic\_1 & 2696.36 & 19    \\
        test\_expectimax\_optimistic\_2 & 12540.77 & 125   \\
        test\_expectimax\_optimistic\_3 & 1183.11 & 14    \\
        test\_expectimax\_pessimistic\_1 & 3736.95 & 26    \\
        test\_expectimax\_pessimistic\_2 & 12692.28 & 129   \\
        test\_expectimax\_pessimistic\_3 & 1089.85 & 16 \\ \bottomrule
    \end{tabular}
    \caption{Résultats avec Expectimax}
\end{table}


    \section{Comparaison}

    On remarque que la performance avec Minimax semble plus stable d'un execution à l'autre, par rapport à Expectimax.
    Avec Expectimax,l'adversaire arrive dans certains cas à augmenter grandement le nombre de coups nécesaire à la résolution de la partie,
    par exemple, c'est le cas de «~test\_expectimax\_optimistic\_2~», ou le nombre de coups est de 125 et le temps d'exécution
    augmenté proportionnellement.
    Pour ce qui est du seuil de vicoire, on peut le fixer à une valeur equivalente à 3 fois la valeur qu'on aurait obtenu au TP 1.
    Lorsque le nombre de coup nécessaires est augmenté par un facteur de 3, on évalue que l'adversaire a bien travaillé.

\end{document}
